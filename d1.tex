\documentclass{l3deliverable}

\usepackage{url}
\usepackage{float}
\restylefloat{table}

\title{Team Organisation}

\author{
    Gordon Reid: 1002536R\\
    Ryan Wells: 1002253W\\
    Kristopher Stewart: 1007175S\\
    David Selkirk: 1003646S\\
    James Gallagher: 0800899G\\
}

\date{27 September 2012}
\deliverableID{D1}
\project{PSD3 Group Exercise 1}
\team{W}

\begin{document}

\maketitle

\section{Introduction}

\subsection{Identification}
This is the Management Plan of the Level 3 Project for Team W.  The project 
relates to the Internship Management System for Software Engineering (SE) and 
Electronic and Software Engineering (ESE) students in the School of Computing 
Science.

\subsection{Related Documentation}
PSD3 Group Exercise Description:\\
\url{fims.moodle.gla.ac.uk/file.php/128/coursework/psd3-ge-1-rev3278.pdf}

\subsection{Purpose and Description of Document}

This document establishes the roles and responsibilities of the members of 
PSD3 Team W in the assigned task of producing an integrated system for 
students to view advertisements for Summer internships proposed by 
organisations that have been cleared as acceptable by the course coordinator, 
and for students studying the Software Engineering (SE) and Electronic and
Software Engineering (ESE) courses to contact and be notified of application 
success by organisations.

Herein the descriptions of the individual responsibilities and deadlines will 
be stated and other necessities will be established for the clarity and 
reference of members of Team W.

\subsection{Document Status and Schedule}

The document will be stored inside a GitHub repository for a variety of reasons:
ease of access for all team members, extensive version control with a commit log
for proof of contribution and its built in Issue system for logging mistakes or 
improvements to the specific person. Table \ref{tab:DeadlineTable} states the 
deadlines and release dates set by the client and these will be achieved 
through weekly meetings and through the methods of communications established 
in section~\ref{sec:com}.

Each member of the team will be responsible for creating 
their own documentation for the part of project they are responsible for and 
this documentation will be collated and normalised by the librarian of this 
project. All non-standardised documentation will be available to the team 
members, but only the standardised version will be publicised. This is to 
ensure all documentation has been peer-reviewed before publication to ensure
a high level of accuracy and coherence. 

\begin{table}[H]
\begin{tabular}{|c|c|c|}
  \hline
  Submission Name & Submission Date & Submission Summary \\
  \hline
  D1 & 10am Thursday 27th September & Group Organisation Description\\
  D2 & 10am Thursday 11th October & Project Plan\\
  D3 & 10am Thursday 1st November & Requirements Specification\\
  \hline
\end{tabular}
\caption{Specified Deadlines}
\label{tab:DeadlineTable}
\end{table}

\pagebreak

\section{Roles}

The team has decided on the following roles:\\
Gordon Reid - Toolsmith. \\
James Gallagher - Secretary. \\
Chief Architect \\
Quality Assurer \\
Test Manager \\
Ryan Wells - Librarian

\section{Authority}

The group has decided to delegate authority by role. Roles are
volunteered for on a temporary basis, allowing for both flexibility
and persistence, as the member or group decides. Roles can be rotated
or adjusted to allow for circumstances such as deadlines or
illness. This approach was chosen to play to the individual members
strengths and facilitate group cohesion, as none of us can claim to be
'senior' to the others.

\section{Communication}
\label{sec:com}

The team is going to utilise three main methods of communication: a
communal Facebook group, face to face meetings, and GitHub.

The methods were chosen to allow information flow at any time or location,
and in various levels of depth. Facebook allows instant communication
of important information since the group are all equipped with
smart phones, while the extensive logging facilities of GitHub let us
leave detailed information on any changes to the current system and
documentation.

We decided early on that face to face meetings were
crucial in the broader aspects of design, since any misunderstandings
or miscommunication at this level could lead to massive problems
further down the line.

\section{Information Management}

All code and deliverables is going to be kept in a central GitHub repository 
with each team member having full access. In addition to this each team member
is going to keep a weekly blog on Mahara. Physical documentation,
including minutes from meetings and any paper design work, will be
stored by the individual under whose responsibilities the
documentation applies (the Secretary will keep the minutes etc).

The team is going to investigate the practicality of digitally backing up
physical documentation, to be stored in the GitHub repository.

\section{Organisational Risks}

The loose structure of the group does have various associated
risks. The lack of concrete structure means the team has to stay on
task, as they have no boss to drive them should they slack off. The
weekly meetings have been implemented to try and offset this, but it
is possible that this may not be as effective as we hope. Furthermore
the changing structure of the roles could lead to confusion over who
is doing what from week to week. We plan to clarify the weeks tasks at
the end of each meeting to counter this issue.

\pagebreak

\appendix

\section{Appendix}

Meeting plan - The group will meet twice a week (once with supervisor, once 
unsupervised) for hour long sessions as follows.\\
\\
Tuesday, 11am. Thursday, 11am. SUBJECT TO CHANGE.
\end{document}
